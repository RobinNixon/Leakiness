\documentclass[11pt,a4paper]{article}

% Packages
\usepackage[utf8]{inputenc}
\usepackage[T1]{fontenc}
\usepackage{amsmath,amssymb,amsfonts}
\usepackage{graphicx}
\usepackage{booktabs}
\usepackage{hyperref}
\usepackage[margin=1in]{geometry}
\usepackage{natbib}
\usepackage{float}
\usepackage{caption}
\usepackage{subcaption}
\usepackage{algorithm}
\usepackage{algpseudocode}
\usepackage{xcolor}

% Title
\title{Substrate Leakiness Predicts Life-Like Behavior: A Two-Axis Framework for Engineering Self-Maintenance in Discrete Dynamical Systems}

\author{Robin Nixon, Author \& AI Researcher}

\date{}

\begin{document}

\maketitle

\begin{abstract}
What determines whether a discrete dynamical system can exhibit life-like behavior---persistent, bounded activity that maintains coherent structures? We present a systematic investigation across five substrate types using \emph{stickiness}, a temporal bit inertia mechanism that requires state changes to be confirmed across multiple timesteps before execution. Our key findings are: (1) A weighted \emph{leakiness} metric combining Lyapunov growth rate, escape dimensions, and branching factor predicts life-like percentage with $R^2 = 0.96$ and perfect rank correlation; (2) Stickiness operates as a temporal low-pass filter rather than spatial selective damping, with acceptance ratio scaling as $1/d$ where $d$ is confirmation depth; (3) The optimal confirmation depth $d^*$ can be predicted from just two Lyapunov measurements via a power-law control formula, with estimation accuracy $r = 0.996$; (4) A compression-based capacity metric separates life-like-prone from resistant substrates with AUC $= 0.944$ on an expanded 50-rule validation set. These results yield a complete engineering protocol: given any discrete substrate, three measurements (activity, compression, two-point Lyapunov calibration) suffice to predict whether life-like behavior will emerge and at what confirmation depth.

\textbf{Keywords:} cellular automata, self-organization, life-like behavior, Lyapunov exponent, algorithmic complexity, temporal filtering
\end{abstract}

\section{Introduction}

The question of what makes a dynamical system capable of supporting life-like behavior---persistent, bounded, self-maintaining activity---has been approached from multiple directions. Wolfram's classification of cellular automata identified Class IV rules as occupying an ``edge of chaos'' between order and randomness \citep{wolfram2002}. Langton's $\lambda$ parameter attempted to predict complexity from rule statistics \citep{langton1990}. More recently, the Universal Complexity Threshold (UCT) identified a 5-bit boundary for universal computation.

Yet these frameworks leave a fundamental question unanswered: given an arbitrary discrete substrate, can we predict \emph{a priori} whether it will support life-like behavior, and if so, what modifications are required to achieve it?

This paper presents a two-axis framework that addresses this question directly. Through systematic investigation across five substrate types---Binary 1D Elementary Cellular Automata (ECA), 2D Cellular Automata, Ternary CA, Discretized Vector Fields, and Semantic Vector systems---we identify two orthogonal axes that together determine life-like potential:

\begin{enumerate}
    \item \textbf{Leakiness} (Axis 1): How readily perturbations escape and amplify. Measured via Lyapunov-like growth rate, escape dimensions, and branching factor. This axis determines \emph{how much} temporal filtering (stickiness) is required.

    \item \textbf{Capacity} (Axis 2): Whether the substrate has sufficient structural complexity to support coherent patterns. Measured via compression ratio. This axis determines \emph{whether} temporal filtering will succeed.
\end{enumerate}

Our central contribution is a validated engineering protocol: measure activity and compression to pre-screen substrates, perform two-point Lyapunov calibration to estimate filter efficiency, then apply the predicted confirmation depth.

\subsection{Definitions}

\textbf{Life-like behavior} requires three properties operating simultaneously:
\begin{itemize}
    \item \emph{Control}: States reliably transition toward configured target patterns
    \item \emph{Stability}: Perturbations do not destroy operational patterns
    \item \emph{Activity}: The system maintains ongoing dynamics (not frozen)
\end{itemize}

\textbf{Stickiness} is a temporal bit inertia mechanism: a cell's state changes only after the underlying rule has consistently requested that change for $d$ consecutive timesteps. The parameter $d$ is the \emph{confirmation depth}.

\textbf{Leakiness} refers to how readily perturbations escape containment and amplify through the system.

\section{Methods}

\subsection{Substrate Definitions}

We investigate five substrate types spanning different dimensionalities and state spaces:

\begin{table}[H]
\centering
\caption{Substrate types investigated}
\begin{tabular}{lcccc}
\toprule
Substrate & Dimensions & States & Size & Rule Type \\
\midrule
Binary 1D ECA & 1 & 2 & 100 & Wolfram rules \\
Binary 2D CA & 2 & 2 & 20$\times$20 & Game of Life \\
Ternary 1D CA & 1 & 3 & 100 & Random 3-state \\
Discretized Vector Field & 1 & 8 & 100 & Direction-voting \\
Semantic Vectors & 1 & $2^k$ & 50 & High-dim embedding \\
\bottomrule
\end{tabular}
\end{table}

\subsection{Leakiness Metrics}

We measure four complementary aspects of perturbation dynamics:

\textbf{Lyapunov-like Growth Rate}:
\begin{equation}
L = \frac{1}{n}\log(\text{dist}(s_n, s'_n))
\end{equation}

\textbf{Escape Dimensions}: Number of independent spatial directions a perturbation can travel.

\textbf{Branching Factor}: Average number of cells affected by a single-cell perturbation after one step.

\textbf{State Channels}: Effective number of extra state values.

The weighted leakiness combines these:
\begin{equation}
\text{Leakiness} = 0.33 \cdot L + 0.29 \cdot E + 0.28 \cdot B + 0.10 \cdot C
\end{equation}

\subsection{Compression Measurement}

We measure algorithmic complexity via compression ratio:
\begin{enumerate}
    \item Generate spacetime diagram: 200 timesteps $\times$ system width
    \item Serialize as raw bytes
    \item Compress with gzip at maximum compression
    \item Compute bits per cell: $\frac{\text{compressed\_size} \times 8}{\text{total\_cells}}$
\end{enumerate}

\section{Results}

\subsection{Leakiness Predicts Life-Like Percentage}

Our weighted leakiness metric achieves near-perfect prediction of life-like behavior (Figure~\ref{fig:leakiness}):

\begin{table}[H]
\centering
\caption{Leakiness and life-like percentage by substrate}
\begin{tabular}{lcc}
\toprule
Substrate & Leakiness & Life-Like \% \\
\midrule
Discretized Vector Field & 0.22 & 100.0\% \\
Binary 1D ECA (Rule 110) & 0.35 & 83.7\% \\
Semantic Vectors & 0.42 & 39.0\% \\
Ternary CA & 0.53 & 36.7\% \\
Binary 2D CA & 0.64 & 17.5\% \\
\bottomrule
\end{tabular}
\end{table}

\textbf{Correlation}: Pearson $r = -0.992$, Spearman $\rho = -1.0$

The relationship follows a sigmoid phase transition:
\begin{equation}
\text{Life-Like \%} = \frac{115}{1 + \exp(6.5 \cdot (L - 0.39))}
\end{equation}

This achieves $R^2 = 0.956$, explaining 95.6\% of variance.

\begin{figure}[H]
\centering
\includegraphics[width=0.8\textwidth]{figures/fig1_leakiness_vs_lifelike.png}
\caption{Weighted leakiness predicts life-like percentage with $R^2 = 0.96$. The sigmoid fit shows a phase transition at $L \approx 0.39$.}
\label{fig:leakiness}
\end{figure}

\subsection{The Mechanism: Temporal Filtering}

Stickiness operates as a \textbf{temporal low-pass filter}, not spatial selective damping:

\begin{table}[H]
\centering
\caption{Temporal filter signature: acceptance ratio $\approx 1/d$}
\begin{tabular}{ccccc}
\toprule
Depth & Attempt Rate & Acceptance & Lyapunov & Activity \\
\midrule
1 & 42.4\% & 100.0\% & +0.111 & 42.4\% \\
2 & 42.1\% & 50.0\% & +0.112 & 21.0\% \\
4 & 41.5\% & 25.0\% & +0.080 & 10.4\% \\
8 & 41.3\% & 12.6\% & +0.034 & 5.2\% \\
12 & 41.3\% & 8.4\% & +0.035 & 3.5\% \\
\bottomrule
\end{tabular}
\end{table}

The attempt rate remains constant ($\sim$42\%) while acceptance scales as $1/d$.

\begin{figure}[H]
\centering
\includegraphics[width=0.9\textwidth]{figures/fig2_temporal_filter_signature.png}
\caption{Stickiness as temporal low-pass filter. Acceptance ratio scales as $1/d$.}
\label{fig:temporal}
\end{figure}

\subsection{Two-Point Calibration Protocol}

The Lyapunov decay follows a power law:
\begin{equation}
L(d) = L_{min} + (L_0 - L_{min}) \cdot d^{-\gamma}
\end{equation}

While $\gamma$ cannot be predicted from baseline properties, it can be estimated from two measurements:
\begin{equation}
\gamma_{est} = \frac{\log((L_0 - L_{min}) / (L_4 - L_{min}))}{\log(4)}
\end{equation}

Correlation between two-point and full-fit: $r = 0.996$.

The control law for optimal depth:
\begin{equation}
d^* = \left(\frac{L_0 - L_{min}}{L_{crit} - L_{min}}\right)^{1/\gamma}
\end{equation}

\begin{figure}[H]
\centering
\includegraphics[width=0.9\textwidth]{figures/fig3_depth_prediction.png}
\caption{Power-law Lyapunov decay and optimal depth prediction.}
\label{fig:depth}
\end{figure}

\subsection{The Capacity Axis: Compression Ratio}

Substrates split into life-like-prone and resistant classes:

\begin{table}[H]
\centering
\caption{Compression separates prone from resistant substrates}
\begin{tabular}{lccc}
\toprule
Substrate & Type & Bits/Cell & Prone \\
\midrule
Rule 110 & complex & 0.95 & YES \\
Rule 54 & complex & 0.88 & YES \\
2D CA & complex & 0.51 & YES \\
Rule 30 & chaotic & 1.33 & NO \\
Rule 90 & additive & 1.33 & NO \\
Rule 150 & additive & 1.33 & NO \\
\bottomrule
\end{tabular}
\end{table}

Threshold: bits/cell $< 1.1 \rightarrow$ PRONE; bits/cell $\geq 1.1 \rightarrow$ RESISTANT

\begin{figure}[H]
\centering
\includegraphics[width=0.9\textwidth]{figures/fig4_two_axis_framework.png}
\caption{Two-axis framework: compression (capacity) vs Lyapunov (leakiness).}
\label{fig:twoaxis}
\end{figure}

\subsection{Validation on Expanded Set}

Validation on 50 ECA rules revealed the need for a two-criterion classifier:
\begin{enumerate}
    \item Activity gate: activity $> 1\%$ OR Lyapunov $> 0$
    \item Compression threshold: bits/cell $< 1.1$
\end{enumerate}

\begin{table}[H]
\centering
\caption{Two-criterion classifier performance}
\begin{tabular}{lcc}
\toprule
Metric & Compression Only & Two-Criterion \\
\midrule
Accuracy & 57.1\% & \textbf{86.7\%} \\
AUC (active rules) & 0.545 & \textbf{0.944} \\
False Positive Rate & 45.5\% & 16.7\% \\
False Negative Rate & 33.3\% & \textbf{0.0\%} \\
\bottomrule
\end{tabular}
\end{table}

\begin{figure}[H]
\centering
\includegraphics[width=0.9\textwidth]{figures/fig5_generalization_validation.png}
\caption{Validation on 50 ECA rules with two-criterion classifier.}
\label{fig:validation}
\end{figure}

\begin{figure}[H]
\centering
\includegraphics[width=0.9\textwidth]{figures/fig6_validation_pooled.png}
\caption{Pooled validation results showing the relationship between predicted and observed life-like behavior across all validation substrates.}
\label{fig:pooled}
\end{figure}

\section{The Complete Engineering Protocol}

\subsection{Decision Tree}

\begin{algorithm}[H]
\caption{Life-Like Prediction Protocol}
\begin{algorithmic}[1]
\State Measure substrate at $d=1$
\If{activity $\leq 1\%$ AND Lyapunov $\leq 0$}
    \State \Return RESISTANT (dead)
\EndIf
\If{compression $\geq 1.1$ bits/cell}
    \State \Return RESISTANT (chaotic)
\EndIf
\State \textbf{Substrate is PRONE}
\State Measure $L_0$ at $d=1$, $L_4$ at $d=4$
\State $\gamma \gets \log((L_0 + 0.05)/(L_4 + 0.05))/\log(4)$
\State $d^* \gets ((L_0 + 0.05)/0.10)^{1/\gamma}$
\State Apply confirmation depth $d^*$
\State \Return Life-like behavior emerges
\end{algorithmic}
\end{algorithm}

\subsection{Measurement Budget}

The complete protocol requires only \textbf{three measurements}:
\begin{enumerate}
    \item Activity (one run at $d=1$)
    \item Compression (same run, computed post-hoc)
    \item Lyapunov at $d=1$ and $d=4$ (for prone substrates only)
\end{enumerate}

\section{Discussion}

\subsection{Why Temporal Filtering Works}

Stickiness blocks transient fluctuations (noise) while preserving sustained dynamics (signal). It preserves 20$\times$ more activity than spatial consensus at comparable Lyapunov reduction.

\subsection{What Compression Captures}

Prone substrates are characterized by \emph{structured novelty}---patterns that are non-random (compressible) but not repetitive. This inverts the ``edge of chaos'' expectation: prone substrates are more structured, not at intermediate complexity.

\subsection{Relationship to Computation}

Computation and life-like behavior appear orthogonal:
\begin{itemize}
    \item Rule 110 is Turing-complete \citep{cook2004} and life-like-prone
    \item Rule 30 is used in cryptographic RNG but life-like-resistant
\end{itemize}

\section{Limitations and Future Work}

\begin{itemize}
    \item Edge cases (Rules 243, 93) suggest gradient rather than hard boundary
    \item $\gamma$ remains an irreducible substrate property
    \item Extension to continuous substrates and biological systems
\end{itemize}

\section{Conclusion}

We have presented a two-axis framework for predicting life-like behavior:
\begin{enumerate}
    \item \textbf{Leakiness} predicts life-like percentage with $R^2 = 0.96$
    \item \textbf{Stickiness} works via temporal filtering
    \item \textbf{Two-point calibration} predicts optimal depth with $r = 0.996$
    \item \textbf{Compression} separates prone from resistant with AUC $= 0.944$
\end{enumerate}

Life-like behavior is not mysterious. It is predictable from substrate properties alone.

\section*{Acknowledgments}

This research was conducted with assistance from Claude (Anthropic), which contributed to experimental design, code implementation, and analysis.

\bibliographystyle{plainnat}
\begin{thebibliography}{9}

\bibitem[Wolfram(2002)]{wolfram2002}
Wolfram, S. (2002).
\newblock \emph{A New Kind of Science}.
\newblock Wolfram Media.

\bibitem[Langton(1990)]{langton1990}
Langton, C.~G. (1990).
\newblock Computation at the edge of chaos: Phase transitions and emergent computation.
\newblock \emph{Physica D}, 42(1-3):12--37.

\bibitem[Cook(2004)]{cook2004}
Cook, M. (2004).
\newblock Universality in elementary cellular automata.
\newblock \emph{Complex Systems}, 15(1):1--40.

\bibitem[Li and Vit{\'a}nyi(2008)]{li2008}
Li, M. and Vit{\'a}nyi, P. (2008).
\newblock \emph{An Introduction to Kolmogorov Complexity and Its Applications}.
\newblock Springer.

\bibitem[Crutchfield and Young(1989)]{crutchfield1989}
Crutchfield, J.~P. and Young, K. (1989).
\newblock Inferring statistical complexity.
\newblock \emph{Physical Review Letters}, 63(2):105.

\end{thebibliography}

\end{document}
